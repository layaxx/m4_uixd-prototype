\subsubsection{Aktivierungsmuster}\label{aktivierungsmuster}

Während des Experiments fielen uns drei Muster auf, wie Passanten auf die Installation aufmerksam wurden.
Bis auf eine einzige Person haben alle interagierenden Passanten gemeinsam, dass sie nur deshalb interagierten, weil sie auf ihrem Weg sowieso an der Installation vorbeigehen mussten und somit zufällig auf die Wahlkabine oder Display 1
aufmerksam wurden.
Nur eine einzige Person querte zielgerichtet die Straße und wich somit von ihrem eigentlichen Weg ab, um zu partizipieren.

\paragraph{Muster 1: Vorbeigehen \& Schauen}
Das häufigste Interaktionsmuster bestand darin, dass Passanten, die den Gehsteig vor dem Stadt:Raum entlangliefen, die Ergebnisse auf Display 1 betrachteten oder die Wahlkabine beim Vorbeigehen kurz in Augenschein nahmen.
Display 1, dass die Wahlumfrageergebnisse zeigte, verleitete damit Passanten dazu, anzuhalten und sich die Ergebnisse einen Moment anzusehen, bevor sie weitergingen.
Die Wahlkabine diente in diesem Fall als Trigger, der die Aufmerksamkeit auf den Stadt:Raum zog und dadurch womöglich die Wahrnehmung von Display 1 förderte.
Nur wenige Passanten blieben stehen, um die Wahlkabine genauer anzusehen, sondern widmeten ihr im Vorbeigehen ein paar Blicke.
Drei Passanten betraten zwar die Wahlkabine, wählten aber nicht.
Der Großteil der Passanten hat natürlicherweise weder Display 1, noch der Wahlkabine ihre Aufmerksamkeit geschenkt.

\paragraph{Muster 2: Vorbeigehen \& Wählen}
Alle Wähler, bis auf Einen, kamen zufällig an der Wahlkabine vorbei und interagierten infolge dessen.
Keiner der Wähler hat nachträglich auf dem benachbarten Display 1 nachgesehen, wie die Verteilung der Gesamtstimmen ausfällt.
Die Wähler waren alle männlich und mehrheitlich vermutlich über 40 Jahre alt.

\paragraph{Muster 3: Straße queren}
Ein Passant (männlich, ca. 18 Jahre alt) querte vor der Interaktion aktiv die Straße und betrachtete auf Display 1 die Gesamtverteilung der Stimmen.
Er fotografierte das Display mit seinem Smartphone und betrat daraufhin die Wahlkabine, um seine Stimme abzugeben.
Da sich die Wahlentscheidung des jungen Manns mit der Partei mit den zu dieser Zeit meisten Stimmen deckte, mutmaßen wir, dass die Umfrageergebnisse ihn dazu motiviert haben, selbst abzustimmen.

