\subsection{Verbesserungspotenzial}\label{verbesserungspotenzial}

Das Experiment hat gezeigt, dass die Installation die beabsichtigten Interaktionen triggern und von Passanten verstanden werden kann.
Es gab jedoch auch Passanten, die nicht wussten, was sie tun sollten.
Einzelne Teilaspekte des Konzepts gingen zudem nicht vollständig auf.
Es stellte sich heraus, dass der wahrnehmbare Zusammenhang der beiden Displays sowie die Displayinhalte nicht optimal gelöst waren.
Auch das User-Feedback an der Wahlkabine ließe sich verbessern.

\subsubsection{Zusammengehörigkeit der Displays}\label{zusammenhang-der-displays}

Während des Experiments hat sich herausgestellt, dass der Zusammenhang der beiden Displays vermutlich nicht für alle ersichtlich war.
Passanten wählten, ohne sich vor oder nach ihrer Wahl das nur wenige Meter entfernte Display mit den Umfrageergebnissen anzusehen.
Gleichermaßen haben Personen, die sich interessiert die Ergebnisse angesehen haben, nicht an der Wahl teilgenommen.

Wir gingen davon aus, dass die Einsicht in die Abstimmungsergebnisse als Trigger dienen kann.
Einerseits, indem man seine Wahlpräferenz unterrepräsentiert sieht und der Partei infolge dessen zu mehr Stimmen verhelfen möchte.
Andererseits, indem man seine Partei gut performen sieht und dementsprechend noch eine weitere Stimme zur Unterstützung gibt.
Ein Mann (ungefähr 50 Jahre alt) schaute im Abstand einiger Zeit sogar zweimal auf die Umfrageergebnisse auf Display 1 und zeigte durch Kopfschütteln eine Reaktion auf die Ergebnisse.
Er gab seine eigene Stimme jedoch nicht ab.

Deshalb hätte auf beiden Displays klarer darauf hingewiesen werden müssen, dass noch ein weiteres Display existiert.
Die Information war zwar in Textform vorhanden, ging aber offenbar auf dem Display unter.
Hierzu sollten die Displayinhalte überarbeitet werden.

\subsubsection{Displayinhalte}\label{displayinhalte}

Die beiden Displays waren weitestgehend statisch und veränderten sich nur dann in Teilen, wenn eine Stimmabgabe registriert wurde.
Statische Displays ziehen bei Weitem nicht so viel Aufmerksamkeit auf sich, wie bewegte Inhalte.
Das wurde uns während des Experiments sehr deutlich klar, da Display 2 (vor der Wahlkabine) durch einen technischen Defekt ungefähr einmal pro Minute kurz aus- und wieder anflackerte, was automatisch den Blick dorthin wandern ließ, wenn sich das Display im Sichtfeld befand.

Anstatt alle Informationen auf einer Seite anzuordnen, hätte man sie dynamisch nacheinander anzeigen lassen können - mit Animationen, die die Aufmerksamkeit auf sich ziehen.
Dadurch wäre es auch möglich gewesen, den Displayzusammenhang zu verdeutlichen.

Für Display 2 in der Wahlkabine stellte sich heraus, dass durch die sehr nahe Platzierung der Wahlkabine am Schaufenster das Sichtfeld eingeschränkt wurde.
Das Display war nicht vollständig wahrnehmbar, ohne seinen Kopf zu bewegen.
Der Blick fiel somit immer nur auf einen Teilbereich des Displays, wohingegen die Bereiche drum herum währenddessen auch aus dem peripheren Sichtfeld verschwanden und damit blind waren.
Zwar testeten wir die Anzeige vorher, doch ohne dazu die Wahlkabine aufzustellen, da diese durch ihre Größe etwas schwierig zu transportieren ist.
Ein Test mit Wahlkabine hätte dieses Problemfeld früher aufzeigen können.

\subsubsection{User-Feedback während der Wahl-Interaktion}\label{user-feedback-bei-der-wahl-interaktion}

Während des Experiments ist aufgefallen, dass Teilnehmer, die das System nicht kennen, beim Wählen nach unten auf die Wahlurne blicken und nicht geradeaus auf das Display.
Ein zusätzliches Feedback als Erfolgsindikator, etwa ein Ton oder ein Leuchtsignal direkt an der Wahlurne, könnte das Verständnis erleichtern.